% Erstelle ein Dokument vom Typ "article"
\documentclass[12pt]{article}

% Informationen über das Dokument
\title{Naturwissenschaften}
\date{30.01.2019}
\author{Fabian Bösiger}

% Für deutsche Sonderzeichen
\usepackage{german}
\usepackage{amsmath}
\usepackage{graphicx}
\usepackage{floatrow}
\usepackage{url}

% Grafische Einstellungen
% Setze Abstand zwischen Paragraphen
\setlength{\parskip}{0.5em}

\begin{document}
	
	\pagenumbering{gobble}
	\maketitle

	\newpage
	\tableofcontents
	
	\newpage
	\pagenumbering{arabic}

	\section{Physiker}

		\begin{table}
			\begin{tabular}{l|l|l}
				\textbf{Name} & \textbf{Geboren am} & \textbf{Gestorben am}\\
				\hline
				Archimedes & 287 v. Chr. & 212 v. Chr \\
				Isaac Newton & 4. Januar 1643 & 31. März 1727
				\caption{Übersicht über die Geburts- und Todesdaten.}
				\label{table:overview}
			\end{tabular}
		\end{table}

		\subsection{Archimedes}
			\begin{figure}
				\includegraphics[width=0.5\linewidth]{images/archimedes.jpg}
				\caption{Archimedes am studieren.}
				\label{image:archimedes}
			\end{figure}
			% Neue Linien werden nicht beachtet
			Archimedes, siehe Grafik \ref{image:archimedes}, war ein griechischer Mathematiker, Physiker und Ingenieur.
			Er gilt als einer der bedeutendsten Mathematiker der Antike. \par
			% Neuer Paragraph mit \par
			Archimedes, geboren ca. 287 v. Chr. wahrscheinlich in der Hafenstadt Syrakus auf Sizilien, war der Sohn des Pheidias, eines Astronomen am Hof Hierons II. von Syrakus. Mit diesem und dessen Sohn und Mitregenten Gelon II. war er befreundet und möglicherweise verwandt. \cite{wikipedia:archimedes}
		
		\subsection{Isaac Newton}
			Isaac Newton hat die Formel \ref{equation:gravity} für die Gravitationskraft gefunden.
			\subsubsection{Formeln}
				% Brüche und tiefgestellte Zeichen
				\begin{equation}
					F = G\frac{m_1m_2}{r^2}
					\label{equation:gravity}
				\end{equation}
				\begin{align}
					f(x) &= x^2 \\
					F(x) &= \int^a_b \frac{1}{3}x^3
				\end{align}
				
		\subsection{Albert Einstein}
			% Gleichungen innerhalb eines Textes
			% Fussnoten
			Albert Einstein war ein cooler Typ, denn er fand die Formel $E = mc^2$.\footnote{\label{footnote:energy-and-mass} Äquivalenz von Masse und Energie}
	
	\section{Mathematiker}

		\subsection{James Joseph Sylvester}
			% Zitate
			Dieser Mann hat die bezeichnung \glqq{}Matrix\grqq{} eingeführt.
			% Zitat in einem Zitat
			Er sagte einfach: \glqq{}Ich führe das Wort \glq{}Matrix\grq{} ein!\grqq{}
			\subsubsection{Beispiele}
				\begin{equation}
					\left(
					\begin{matrix}
						1 & 0 & \sqrt{2} \\
						0 & 1 & 5 \\
						6 & \pi & 7
					\end{matrix}
					\right)
					\begin{matrix}
						x_1 \\
						x_2 \\
						x_3
					\end{matrix}
				\end{equation}
	
	\newpage
	\begin{appendix}
		\pagenumbering{gobble}
		\bibliography{bibliography} 
		\bibliographystyle{ieeetr}
		\listoffigures
		\listoftables
	\end{appendix}

				
\end{document}