\documentclass{article}

\usepackage{biblatex}
\usepackage{listings}
\usepackage{xcolor}

\addbibresource{viper.bib}
\addbibresource{silicon.bib}

\parskip=1em
\parindent=0pt

\definecolor{codegray}{rgb}{0.95,0.95,0.95}

\lstdefinestyle{mystyle}{
    backgroundcolor=\color{codegray},
    captionpos=b,
    numbers=left,
}
\lstset{style=mystyle}

\title{Avoiding Structural Equality Checks on Immutable Syntax Trees}
\author{
    Fabian Bösiger\\
    Supervised by Dr. Malte Schwerhoff
}
\date{\today}

\begin{document}

    \pagenumbering{gobble}
    \maketitle
    \newpage
    \pagenumbering{arabic}

    \section{Introduction} \label{introduction}
    Viper \cite{viper} is a verification infrastructure on top of which verification tools
    for different programming languages can be built. Silicon \cite{silicon} is a backend for Viper,
    which is based on symbolic execution.
    
    To advance program verification in practice, fast verification is cruicial as
    it profides a more streamlined experience for developers. This is the reason
    why one of Silicons stated goals is performance:

    \begin{quotation}
        The verifier should enable an IDE-like experience: it should be
        sufficiently fast such that users can continuously work on verifying
        programs [...] \cite{silicon}
    \end{quotation}

    Because Silicon is a shared backend for many verification frontends, faster
    verification in Silicon is directly beneficial for the developer using
    such tools like Gobra for Go, Nagini for Python or Prusti for Rust.

    Silicon internally builds an abstract syntax tree (AST) from the Viper input
    program. This AST is checked for structural equality multiple times within
    the execution of Silicon. Currently, Silicon recursively checks for structural
    equality of ASTs. For big ASTs and many equality checks, this process of becomes
    very inefficient.

    \begin{lstlisting}[language=Scala, caption=Example of a term equality check.]
TODO
class Plus(val p0: Term, val p1: Term) extends Term
    
object Plus {
    def apply(e0: Term, e1: Term) = new Plus(e0, e1)
}
    \end{lstlisting}
    
    We can't avoid equality checks themselves, but what we can do is implementing
    equality checks in a more performant way. This is the core topic of this thesis.
    
    Currently, every time a term is applied, a new instance of this term is created.
    This happens even if we apply the same term twice, leaving us with no other choice
    than checking equality structurally and recursively. 

    \begin{lstlisting}[language=Scala, caption=How term instances currently are created.]
class Plus(val p0: Term, val p1: Term) extends Term
    
object Plus {
    def apply(e0: Term, e1: Term) = new Plus(e0, e1)
}
    \end{lstlisting}

    We see that everytime \texttt{apply} is called, a new instance of
    \texttt{class Plus} is created. We want to avoid creating a new instance on
    every call to \texttt{apply} by using the flyweight pattern.
  
    \section{Approach}
    In a first step, we want to analyze the performance of the Silicon backend,
    which allows us in the future to relate possible performance improvements to 
    changes made during the project. We use existing benchmarking infrastructure 
    and, if necessary, create new benchmarks.

    The general idea is to utilize the flyweight pattern to the AST which is immutable.
    To do this, we maintain a pool of pairwise structurally unequal terms. Whenever a term
    is applied, we first compare the new term that is to be created with our
    pool of existing terms. If a structurally equivalent term already exists, we
    avoid creating a new instance of this term and instead return a reference to the 
    equivalent object in the pool.

    This gives us the guarantee that there are no two instances of the
    same term, i.e. every two structurally equivalent terms point to the same
    underlying object in memory. Comparing terms for structural equality boils
    down to a cheap pointer comparison, recursive equality checks can be avoided.

    \begin{lstlisting}[language=Scala, caption=How a memory pool may be used to avoid creating structurally equivalent terms.]
class Plus(val p0: Term, val p1: Term) extends Term
    
object Plus {
    def apply(e0: Term, e1: Term) = {
        val plus = new Plus(e0, e1)
        if (pool.contains(plus))
            pool.get(plus)
        else
            plus
    }
}
    \end{lstlisting}

The Scala compiler automatically generates the \texttt{equals} method for case classes,
where the default implementation recursively checks each field for equality. To avoid
this process of recursive equality checking, we instead override the default \texttt{equals}
method to do a simple reference equality check.

    \begin{lstlisting}[language=Scala, caption=We override the the default equals method to do a simple reference equality check.]
sealed trait Term extends Node {
    // Override the default equals method:
    override def equals(other: Any) =
        this.eq(other.asInstanceOf[AnyRef]) 
    // ...
}
    \end{lstlisting}

    \section{Goals}
    \subsection{Core Goals}
    \begin{enumerate}
        \item \textbf{Research other potential solutions} that may have been
        tried for similar problems. The general problem of recursive equality
        checks may have been solved in the past.
        \\\textit{1 Week}
        \item \textbf{Implement the solution approach} described in section
        \ref{introduction} without the use of macros. This allows for a first
        evaluation of performance before auto-generating code using macro
        annotations.
        \\\textit{3 Weeks}
        \item \textbf{Evaluate performance improvments} and discuss
        the impact of our changes on the time needed for verification.
        \\\textit{2 Weeks}
        \item \textbf{Build macro annotations} for automatic code generation.
        This allows for easy implementation of terms that may be needed in the
        future. Macro annotations also help keeping the codebase well-arranged. 
        \\\textit{4 Weeks}
        \item \textbf{Add IDE support} for our macro annotations. Because
        Scala macros are still in an experimental phase, we make sure that
        IntelliJ picks up code generated by our macro annotations.
        \\\textit{4 Weeks}
    \end{enumerate}
    \subsection{Extension Goals}
    \begin{enumerate}
        \item \textbf{Utilizing better suited data structures} to possibly achieve
        further performance improvments by in Silicon. If the performance of
        equality checks was improved in the previous steps, some data
        structures may become preferrable performance-wise, e.g. hashmaps become
        more performant over a simple iteration over a list.
        \item \textbf{Evaluate further performance improvments} that may be
        achieved in the previous step.
        \item \textbf{Research ways to extend AST simplifications} as a furter step
        to increase performance. Not only local simplifications, but also global
        simplifications are imaginable.
        \item \textbf{Implement better AST simplifications} found in the previous step.
        \item \textbf{Evaluate further performance improvments} that may be
        achieved in the previous step.
        \item \textbf{Use a DSL} in combination with Scala macros to
        auto-generate AST simplifications. This would allow to easily add or modify
        AST simplifications in the future.

    \end{enumerate}

    \printbibliography
    
\end{document}