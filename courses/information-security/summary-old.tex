% !TEX program = xelatex

\documentclass[11pt]{article}

\usepackage{listings}
\usepackage{xcolor}
\usepackage{fontspec}
\usepackage{graphicx}
\usepackage{amsmath}
\usepackage{amssymb}

\DeclareMathOperator{\Enc}{Enc}
\DeclareMathOperator{\Dec}{Dec}
\DeclareMathOperator{\xor}{xor}
\DeclareMathOperator{\Tag}{Tag}
\DeclareMathOperator{\Vrfy}{Vrfy}


\setmonofont[
  Contextuals={Alternate},
  Scale=0.8
]{Fira Code}

\parskip=1em
\parindent=0pt

\definecolor{codegrey}{rgb}{0.95,0.95,0.95}
\definecolor{commentgrey}{rgb}{0.5,0.5,0.5}

\title{Information Security Summary}
\author{Fabian Bösiger}

\lstdefinestyle{mystyle}{
    basicstyle=\ttfamily,
    keywordstyle=\bfseries,
    commentstyle=\color{commentgrey},
    backgroundcolor=\color{codegrey},
    frame=single,
    numbers=left,
}
\lstset{style=mystyle}

\begin{document}

    \pagenumbering{gobble}
    \maketitle
    \newpage

    \tableofcontents
    \newpage
    \pagenumbering{arabic}

    \section{Introduction}
    
    Key space: $\mathcal{K}$\\
    Plaintext space: $\mathcal{M}$\\
    Ciphertext space: $\mathcal{C}$\\
    Encryption algorithm: $\Enc_k(m): \mathcal{K} \times \mathcal{M} \to \mathcal{C}$\\
    Decryption algorithm: $\Dec_k(c): \mathcal{K} \times \mathcal{C} \to \mathcal{M}$\\
    Encryption scheme: $(\Enc_k, \Dec_k)$
    Correctness: $\forall k: \Dec_k(\Enc_k(m)) = m$

    \section{Historical Ciphers}

    \subsection{Caesars's Shift Cipher}

    $\mathcal{M} = \{A, \dotsc, Z\} = \{0, \dotsc, 25\}$\\
    $\mathcal{K} = \{0, \dotsc, 25\}$\\
    $\Enc_k(m_0, \dotsc, m_n) = (m_0 + k \mod 25, \dotsc, m_n + k \mod 25)$\\
    $\Dec_k(c_0, \dotsc, c_n) = (c_0 - k \mod 25, \dotsc, c_n - k \mod 25)$
    
    \subsubsection{Vulnerabilities}

    Brute force attack.

    \subsection{Substitution Cipher}

    $\mathcal{M} = \{A, \dotsc, Z\} = \{0, \dotsc, 25\}$\\
    $\mathcal{K} = \{0, \dotsc, 25\}$\\
    $\Enc_k(m_0, \dotsc, m_n) = (\pi(m_0), \dotsc, \pi(m_n))$\\
    $\Dec_k(c_0, \dotsc, c_n) = (\pi^{-1}(c_0), \dotsc, \pi^{-1}(c_n))$

    \subsubsection{Vulnerabilities}

    Statistical patterns of the language.

    \subsection{Vigenere Cipher}

    TODO

    \subsubsection{Vulnerabilities}

    \section{Information-Theoretic Security}

    If the key $k$ is chosen randomly and $c := \Enc_k(m)$ is given
    to the adversary, the adversary should not learn any additional
    information about the plaintext $m$.

    An encryption scheme is perfectly secret if for some random variables $M$, $C$
    and every $m$, $c$: $P(M = m) = P(M = m \mid C = c)$.
    
    Equivalently: $M$ and $C$ are independet.

    Equivalently: The distribution of $C$ does not depend on $M$.
    
    Equivalently: For every $m_0$, $m_1$ we have that $\Enc(k, m_0)$ and
    $\Enc(K, m_1)$ have the same distribution.

    In every perfectly secret encryption scheme, we have $|\mathcal{K}| \geq |\mathcal{M}|$.

    \subsection{One-Time Pad}

    $\mathcal{M} = \mathcal{K} = \{0, 1\}^t$\\
    $\Enc_k(m) = k \xor m$\\
    $\Dec_k(c) = k \xor c$

    \subsubsection{Correctness}

    $\Dec_k(\Enc_k(m)) = k \xor (k \xor m)$

    \subsubsection{Perfect Secrecy}

    \begin{align}
        P(C = c \mid M = m)\\
        = P(M \xor K = c | M = m)\\
        = P(m \xor K = c)\\
        = P(K = m \xor c)\\
        = 2^{-t}\\
        = P(C = c \mid M = m_0) = P(C = c \mid M = m_1)
    \end{align}

    \subsubsection{Vulnerabilities}

    Perfectly secret. But the key is as long as the message and cannot be reused.

    \section{Computational Security}

    A system $X$ is $(t, \epsilon)$-secure if every Turing Machine
    that operates in time $t$ can break $X$ with probability of at most $\epsilon$.

    A function $\mu: \mathbb{N} \to \mathbb{R}$ is negligible, if for every natural number
    $c$ there exists $n_0$ such that for all $x > n_0$: $|\mu(x)| < \frac{1}{x^c}$ 

    $M$ and $C$ are independet from the point of view of a
    computationally limited adversary with high probability.

    More formally: $X$ is secure if for all probabilistic poly-time turing machines $M$,
    $P(M \textrm{ breaks the scheme } X)$ is negligible.

    Equivalently: No poly-time adversary can distinguish the distributions
    $\Enc(K, m_0) = \Enc(K, m_1)$ with non-negligible probability.

    \subsection{Chosen-Plaintext Attack}

    Learning phase: Adversary can repeatedly send message $m$ that is encrypted using some
    unknown $k$ and receives $c = \Enc(k, m)$.

    Challenge phase: Adversary sends $m_0$ and $m_1$, receives $c = \Enc(k, m_b)$ for some 
    unknown $b$, has to guess b.

    CPA-security: Every randomized poly-time adversary guesses $b$ correctly with
    probability of at most $\frac{1}{2} + \epsilon(n)$ where $\epsilon$ is negligible.

    CPA-secure encryptions have to be randomized or have a state.

    If a CPA-secure encryption exists with $|k| \leq |m|$, then $P \neq NP$.

    \section{Pseudorandom Functions}

    Select random permutation $F: \{0, 1\}^m \to  \{0, 1\}^m$, give it to both parties
    similar to secret key.

    Problem: $F$ requires $m * 2^m$ space.

    Solution: Pseudorandom functions using a key $F_k: \{0, 1\}^* \times \{0, 1\}^* \to \{0, 1\}^*$.

    A keyed permutation $F_k$ is pseudorandom if it cannot be distinguished from a completely
    random function. More formally assume two scenarios where a distinguisher $D$ tries to distinguish
    random from pseudorandom function:

    Scenario 0: $D$ sends $t$ random messages which are encrypted using the same pseudorandom function $F_k$
    with random keys.

    Scenario 1: $D$ sends $t$ random messages which are encrypted using a true random function $F$.

    $F_k$ is a pseudorandom function if all probabilistic poly-time distinguishers $D$ cannot distinguish
    scenarios 0 and 1 with a non-negligible advantage.

    If a distinguisher additionaly has access to the inverted function $F$, we get the definition of a strong
    pseudorandom function.

    \section{Block Ciphers}

    Block ciphers are pseudorandom permutations $F_k$. THey use a key of $K$ bits to specify a random
    subset of $2^K$ mappings. If the section of mappings is random, the resulting cypher
    will be a good approximation of the ideal block cypher.

    \subsection{Shannon's Confusion and Diffusion Principle}

    Diffusion: Ciphertext bits should depend on the plaintext bits in a complex way.
    If a plaintext bit is changed, ciphertext bits should change with $p=\frac{1}{2}$.

    Confusion: Each bit of the ciphertext should depend on the whole key.
    If one bit of the key is changed, the ciphertext should change entirely.

    \subsection{Confusion-Diffusion Paradigm}

    Confusion: Implement large $F_k(m)$ using smaller $f_i(k, m_i)$, called substitution boxes.
    $F_k(m_1 m_2 \dotsc m_n) = f_1(k, m_1) f_2(k, m_2) \dotsm f_n(k, m_n)$.

    Diffusion: Permute (Mix) the output $F_k$.

    Key idea: Run the confusion and diffusion multiple times.

    \subsection{Data Encryption Standard (DES)}

    $\textrm{Input} \to \textrm{Initial Permutation } IP \to \textrm{Feistel Network depending on } k \to \textrm{Final Permutation } IP^{-1} \to \textrm{Output}$

    TODO

    A 3-round feistel network is a pseudorandom permutation.

    A 4-round feistel network is a strong pseudorandom permutation.

    To fully describe a feistel network we need to describe a key schedule algoritm and the pseudorandom permutation function $f$.

    \subsubsection{Vulnerabilities}

    Key is too short, brute force attack is possible.

    Unclear role of the NSA in the design.

    \subsubsection{Triple Encryption}

    \subsection{Advanced Encryption Standard (AES)}

    TODO: Topic 2

    \section{Stream Ciphers}

    Pseudorandom generators used in practice are called stream ciphers.

    \subsection{Pseudorandom Generators}

    \subsection{RC4}

    \subsubsection{Vulnerabilities}

    Some bytes of the output are biased.

    The first few bytes sometimes leak information about the key.

    \subsection{ChaCha}

    \section{Hash Functions and MACs}

    \subsection{Message Authentication Codes (MACs)}

    Key space: $\mathcal{K}$\\
    Plaintext space: $\mathcal{M}$\\
    Set of tags: $\mathcal{T}$\\
    Tagging algorithm: $\Tag: \mathcal{K} \times \mathcal{M} \to \mathcal{T}$\\
    Verification algorithm: $\Vrfy: \mathcal{K} \times \mathcal{M} \times \mathcal{T} \to \{0, 1\}$\\
    MAC scheme: $(\Tag, \Vrfy)$

    We say that  $(\Tag, \Vrfy)$ is secure, if for all poly-time adversaries $A$,
    $P(A \textbf{ breaks } (\Tag, \Vrfy)) = \epsilon(n)$

\end{document}